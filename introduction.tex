\chapter{INTRODUCTION}

With rapid growth of deep sub-micron technology, there has been an aggressive shrinking in physical dimension of silicon structures that can be realized on silicon. This advancement has enabled the transition of multi million gate designs from large printed circuit boards to SoC (System on Chip). SoC design has the advantages of smaller size, lower power consumption, reliability, performance improvement and low cost per gate. Over the past few years, major challenges faced by semiconductor industry has been to develop more complex SoCs with greater functionality and diversity with reduction in time-to-market. One of the main challenge among this is verification.

Functional verification comprises a large portion of the resources required to design and validate a complex system. Often, the validation must be comprehensive without redundant effort. To minimize wasted effort, coverage is used as a guide for directing verification resources by identifying tested and untested portions of the design. Coverage is defined as the percentage of verification objectives that have been met. It is used as a metric for evaluating the progress of a verification project in order to reduce the number of simulation cycles spent in verifying a design. Coverage collection and management is an important requirement to evaluate the verification progress.

There are two types of coverage metrics: those that can be automatically extracted from the design code, such as code coverage, and those that are user-specified in order to tie the verification environment to the design intent or functionality. The latter form is referred to as functional coverage. Functional coverage is a user-defined metric that measures how much of the design specification, as enumerated by features in the test plan, has been exercised. It can be used to measure whether interesting scenarios, corner cases, specification invariants, or other applicable design conditions, captured as features of the test plan have been observed, validated, and tested. The key aspects of functional coverage are: it is user-specified and is not automatically inferred from the design, it is based on the design specification and is thus independent of the actual design code or its structure.



\section{ORGANIZATION OF THE THESIS}
The organization of this thesis is as follows:\\
\noindent
    {\bf Chapter}~\ref{chap:amd64} gives brief introduction to AMD64 architecture.\\
    {\bf Chapter}~\ref{chap:func_verif} gives brief information on functional verification .\\
    {\bf Chapter}~\ref{chap:coverage} explains about the existing coverage collection methods.\\
    {\bf Chapter}~\ref{chap:coverage_on_emulator} discusses about the proposed enhancements.\\
    {\bf Chapter}~\ref{chap:results} gives final results of the proposed method.\\
    {\bf Chapter}~\ref{chap:conclusion} discusses the various conclusion drawn from the results and the scope for future work.
